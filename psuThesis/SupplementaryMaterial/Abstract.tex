Conventionally, Path Integral Monte Carlo (PIMC) calculations are performed with `classical beads' (beads interacting via a classical potential) by using the primitive approximation for the thermal density matrix. Higher order propagators of the thermal density matrix have been proven to achieve faster convergence and better precision in quantum calculations than using just the primitive approximation. Use of different propagators in PIMC  leads to methods equivalent to performing PIMC with `semi-classical beads' (beads interacting via a semi-classical potential). We examine the Takahashi-Imada (TI) propagator as well as an \emph{ad hoc} semi-classical potential in PIMC calculations for computing the quantum second virial coefficient for helium-4. We compare the performance of the two approaches based on semi-classical beads against values computed from PIMC using conventional classical beads. We find that while the TI propagator has the same or marginally better precision compared to the classical case, it has the best convergence rate (with respect to number of path-integral beads) among the three approaches. The convergence rate of the \emph{ad hoc} potential is marginally better than its classical counterpart, and its precision is approximately the same as the classical case.

We develop new orientation and bond length sampling algorithms for diatomic molecules that involve simple approximations to the probability distributions. Fundamental to both these algorithms is the idea of treating the diatomic molecule as two independent atoms as opposed to one (quantum) rigid rotor. This saves a lot of time by avoiding any quantum chemistry calculations that involve rotational and/or vibrational states. In addition, it also enables us to analyze the sampling problem more mathematically without having to worry about the physics of the rigid rotor. The generality of this idea makes it straightforward (although not trivial) for us to extend it to multiatomic systems which, we expect, will be considerably difficult using (quantum) rigid rotors. We have applied these algorithms in combination with some of the latest \emph{ab initio} potentials (rigid and flexible) of H$_2$ to compute fully quantum second virial coefficients using Path Integral Monte Carlo(PIMC) method.  We observe very good agreement with literature (both experimental and simulation) data. Due to the nature of the approximations present, the performance of the algorithms decrease with increasing $P$ (number of beads). However, we observe no clear trend of the performance of the algorithms with increasing temperature.

We report new, fully quantum virial coefficients excluding exchange effects for nitrogen (up to 3${\rm ^{rd}}$ order) and oxygen (2${\rm ^{nd}}$ order) dimers using Path Integral Monte Carlo (PIMC) method. Additionally, we also report fully quantum virial coefficients calculated using PIMC with semi-classical beads (cite:helium) for the nitrogen dimer (up to 3${\rm ^{rd}}$ order). \AbInitio{} potentials used for this work include those developed by Hellmann \cite{Hellmann2013} and Bartolomei et al. \cite{Bartolomei2010} for nitrogen and oxygen respectively. Our recently developed orientation sampling algorithm (cite: hydrogen) was used in conjunction with Mayer sampling \cite{Singh2004} to compute precise virial coefficients. We investigated 96 temperatures in the range [50 K, 3000 K] for nitrogen and 15 temperatures in the range [90 K, 1400 K] for oxygen.
