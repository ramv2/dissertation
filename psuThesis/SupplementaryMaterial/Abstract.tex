Quantum virial coefficients can be computed using Path Integral Monte Carlo (PIMC) method to include nuclear quantum effects. Developed by Richard Feynman \cite{Feynman}, the path integral formalism essentially maps the quantum partition function of a molecule onto the classical partition function of a fictitious closed ring polymer made up of $P$ ``beads'' for each atom of the molecule. Adjacent beads within a fictitious ring interact via a harmonic potential (depending on the mass $m$, temperature $T$ and the number of beads $P$) while corresponding beads of each ring interact via the given intermolecular potential. The formalism draws parallels between the kernel of the wavefunction from quantum mechanics, and the thermal density matrix related to the partition function from statistical mechanics. By using what is known as the ``primitive'' approximation, the kinetic and potential energy operators of the Hamiltonian are assumed to be commutative. Note that using this approximation the kinetic energy operator leads to the harmonic potential experienced by the adjacent beads and the potential energy operator does not modify the input interaction potential. However, if we use higher order propagators of the Hamiltonian, the potential energy operator transforms the input potential into a form that more resembles a semi-classical potential, i.e., a potential dependent on $\hbar$ and $P$. For ease of calculation, we will restrict our attention to propagtors in which the kinetic energy operator remains the same as that for the primitive approximation. Additionally, such propagators have been proven to achieve faster convergence and better precision in quantum calculations. 

Firstly, we examine the Takahashi-Imada \cite{Takahashi1984} (TI) propagator as well as an \emph{ad hoc} semi-classical potential in PIMC calculations for computing the quantum second virial coefficient for helium-4 using Mayer Sampling Monte Carlo \cite{Singh2004} (MSMC) method. We compare the performance of the two approaches based on semi-classical beads against values computed from PIMC using conventional classical beads. We find that while the TI propagator has the same or marginally better precision compared to the classical case, it has the best convergence rate (with respect to number of path-integral beads) among the three approaches. The convergence rate of the \emph{ad hoc} potential is marginally better than its classical counterpart, and its precision is approximately the same as the classical case.

Going from monatomic to diatomic molecules, rotational and vibrational degrees of freedom add more complexity to the kinetic energy operator and the sampling problem. Conventionally, diatomic molecules have been studied under the quantum rigid rotor approximation for which the kinetic energy operator is fairly well established. However, it is difficult to advance to a full path-integral treatment of rotation and vibration while remaining in the framework of rovibrational quantum states for the diatomic, and even if this were tractable, it does not provide a viable route to extending to multiatomic (more than 2-atom) molecules. A step toward overcoming this problem was made by Garberoglio et al. \cite{Garberoglio2014} in their path-integral treatment of the flexible diatomic. They describe the diatomic molecule by using two independent (albeit bonded) atoms instead of one quantum rigid rotor. This saves a lot of time by avoiding any quantum chemistry calculations that involve rotational and/or vibrational states. In addition, it also enables us to analyze the sampling problem more mathematically without having to worry about the physics of the rigid rotor. 

Secondly, we extend this idea and develop orientation and bond length sampling algorithms for diatomic molecules that involve simple approximations to the probability distributions. The generality of this idea makes it straightforward (although not trivial) for us to extend it to multiatomic systems which, we expect, will be considerably difficult using (quantum) rigid rotors. We have applied these algorithms in combination with some of the latest \emph{ab initio} potentials (rigid and flexible) of H$_2$ to compute fully quantum second virial coefficients using PIMC and MSMC methods. We were able to satisfactorily reproduce the most accurate literature data of Garberoglio et al. \cite{Garberoglio2014}. Due to the nature of the approximations present, the performance of the algorithms decrease with increasing $P$ (number of beads). However, we observe no clear trend of the performance of the algorithms with increasing temperature.

Thirdly, to further verify the robustness of the orientation algorithm, we report new, fully quantum virial coefficients excluding exchange effects for nitrogen (up to 3${\rm ^{rd}}$ order) and oxygen (2${\rm ^{nd}}$ order) dimers using PIMC and MSMC methods. Additionally, we also report fully quantum virial coefficients calculated using PIMC with semi-classical beads for the nitrogen dimer (up to 3${\rm ^{rd}}$ order). \AbInitio{} potentials used for this work include those developed by Hellmann \cite{Hellmann2013} and Bartolomei et al. \cite{Bartolomei2010} for nitrogen and oxygen respectively. We investigated 96 temperatures in the range [50 K, 3000 K] for nitrogen and 15 temperatures in the range [90 K, 1400 K] for oxygen. We observed over all good agreement of our PIMC results with most accurate semi-classical results reported in literature for both these diatomics.

Finally, we report new semi-classical third virial coefficients including flexible contributions for H$_2$O using MSMC method. We observe good agreement with most accurate classical values in literature for the range of temperatures considered.
