Virial coefficients are unique thermodynamic properties of a system owing to their link between interactions at the molecular level to macroscopic quantities such as the pressure. In this work, we take advantage of this feature and compute virial coefficients of a variety of systems by performing simulation studies. The nature and quality of the interaction potential used in such studies highly affects the quality of the resulting virial coefficients. Therefore, we have employed \abinitio{} based interaction potentials that are state-of-the-art and have been developed using high quality quantum chemistry calculations. Naturally, the complexity of such simulations is a strong motivator for the development of algorithms that are highly efficient and yield precise results. In this regard, we have developed two efficient and novel algorithms for use in Path Integral Monte Carlo (PIMC), a method used to incorporate nuclear quantum effects in virial coefficient calculations for diatomic molecules. We have successfully applied these algorithms to compute virial coefficients including quantum effects or, in short, quantum virial coefficients, for H$_2$, N$_2$ and O$_2$ systems. In addition to applying these algorithms to study diatomic molecules, we have also investigated other algorithms like PIMC using semi-classical beads and compared them to conventional PIMC, for He as well as N$_2$ systems. Finally, we have also evaluated virial coefficients including quantum corrections, or, in short, semi-classical virial coefficients for a latest \abinitio{} potential of water.
