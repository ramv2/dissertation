\chapter{Hydrogen}
    \section{Motivation}
        Hydrogen is an extremely important ingredient in many industries including fertilizers, petroleum, aerospace as rocket fuel and energy as an alternative fuel source \cite{Jacobsen2007}. Additionally, the properties of mixtures of hydrogen and other gases are also important in several other fields ranging from metrology (H$_{\rm 2}$ + H$_{\rm 2}$O) \cite{Hodges2004} to astrophysics (H$_{\rm 2}$ + He) \cite{Boothroyd2002,Boothroyd2003} to superfluidics \cite{Patkowski2008,Grebenev2000}. Therefore, computing really precise and accurate physical properties of hydrgoen is vital to several industries as well as research institutions.

        Historically, hydrogen has been, and still is, one of the most studied compounds in quantum chemistry owing to its extemely simple electronic structure comprising of just one electron. Molecular hydrogen exists in two forms with significantly different physical properties \cite{Jacobsen2007}: 1) parahydrogen with anti-parallel nuclear spins and 2) orthohydrogen with parallel nuclear spins. Hydrogen also has two isotopes Deuterium (D) and Tritium (T) with 1 and 2 neutrons respectively. Several studies \cite{Goodwin1963,Kolos1986,Schwenke1988,Mielke2002,Manzhos2010,Garberoglio2010,Garberoglio2012,Sakoda2012,Garberoglio2013,Garberoglio2014} (both computational as well as experimental) of different combinations of the hydrogen molecule and its isotopes ($\allowbreak {\rm H_2, H_3, (H_2)_2, (H_2)_3, HD, HT, D_2, T_2}$ etc.) have been performed to gain further insight into the nature of interactions of these systems. However, for the purpose of evaluating fully quantum virial coefficients using \abInitio potentials, we restrict our attention primarily to the hydrogen dimer ${\rm (H_2)_2}$ and consider both the rigid as well as flexible monomer cases.

        %From the early days of quantum mechanics, scientists have employed hydrogen and its different forms and isotopes not only as test cases for supporting their hypotheses and calculations, but also as building blocks for analyzing systems with more complicated electronic structure. This in turn has contributed to the advancement of computational quantum chemistry in terms of development of new basis sets, theories and approximations for performing improved \abInitio calculations.

    \section{Recent Developments}
        In this section, we give an overview of some the \abInitio PESs that have been developed for studing the hydrogen dimer and in the interest of brevity, we restrict our discussion to the PESs that were developed on or after the year 2000.\\
 
        Diep and Johnson's \cite{Diep2000} pioneering work lead not only to the development of the PIMC method but also two \abInitio PESs for the dimer with rigid monomers. One using the vibrationally-averaged bond length of the dimer and anohter using a slightly smaller equilibrium value, in order to study the effect of bond lengths on the PESs. 37 unique angular configurations were used in the constuction of these PESs (and their corresponding spherical harmonics fits) and electron correlation effects were accounted at the MP2, MP3, MP4 and CCDS(T) levels of theory. They observed excellent agreement of their calculated second virial coefficient values with experiments for T = 15 to 500K. They also concluded that the differences observed in the PESs based on different bond lengths were small. Boothroyd et al. \cite{Boothroyd2002} reported a rigid monomer PES (Ad) that was fit to \Sim 48 000 \abInitio energies out of which \Sim 42 000 were computed using the MRD-CI program of Buenker and Peyerimhoff \hl{include other papers as well} \cite{Buenker1974}. They observed significant improvement over previous PESs in terms of its rms error. Robert J. Hinde \cite{Hinde2008} reported a six dimensional PES that included monomer flexibility as well and used it for studying the IR and Raman transition energies. Patkowski et al. \cite{Patkowski2008} reported a rigid monomer PES that was computed at the CCSD(T) level of theory using very large orbital basis sets. They reported second virial coefficients that were in better agreement with experiments than previous PESs. It was also found that including quantum effects becomes important at \Sim 200K for PIMC calculations of virial coefficients. Most recently, Van and Deiters \cite{Tat2015} have reported a new \abInitio PES at the CCSD(T) level of theory and observed very good agreement of their calculated virial coefficients with experiemental results (where available). 
    \section{Computational details}
    \section{Results and discussion}
