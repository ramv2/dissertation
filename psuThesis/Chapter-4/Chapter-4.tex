\chapter{Hydrogen}
\section{Motivation \& Recent Developments}
    Hydrogen is an extremely important ingredient in many industries including fertilizers, petroleum, aerospace as rocket fuel and energy as an alternative fuel source \cite{Jacobsen2007}. Additionally, the properties of mixtures of hydrogen and other gases are also important in several other fields ranging from metrology (H$_{\rm 2}$ + H$_{\rm 2}$O) \cite{Hodges2004} to astrophysics (H$_{\rm 2}$ + He) \cite{Boothroyd2003}. Therefore, computing really precise and accurate physical properties of hydrgoen is vital to several industries as well as research institutions. 

    Historically, hydrogen is one of the most studied compounds in quantum chemistry owing to its extemely simple electronic structure comprising of just one proton and one electron. Molecular hydrogen exists in two forms with significantly different physical properties \cite{Jacobsen2007}: 1) parahydrogen with anti-parallel nuclear spins and 2) orthohydrogen with parallel nuclear spins. Hydrogen also has two isotopes Deuterium (D) and Tritium (T) with 1 and 2 neutrons respectively. From the early days of quantum mechanics, scientists have employed hydrogen and its different forms and isotopes not only as test cases for supporting their hypotheses and calculations, but also as building blocks for analyzing systems with more complicated electronic structure. This in turn has contributed to the advancement of computational quantum chemistry in terms of development of new basis sets, theories, approximations and other calculation techniques.
     
\section{Computational details}
\section{Results and discussion}
