\chapter{Helium}
\section{Recent developments}
    Calculation of very precise physical properties of helium is of interest in the field of metrology to develop accurate calibration and pressure standards, to accurately compute the Boltzmann constant, and to improve acoustic gas thermometry  \cite{Garberoglio2009,Fellmuth2006,Schmidt2007,Pitre2006,Moldover2010,Aziz1995}. Semi-classical virial coefficients up to fifth order have been computed for helium-4 by Shaul et al. \cite{Shaul2012SC}, and showed that first-principles properties could be evaluated with precision and accuracy that exceeds experiment. Garberoglio and Harvey \cite{Garberoglio2009,Garberoglio2011,Garberoglio2011b} reported fully quantum second and third virial coefficients for helium-3 and helium-4 including exchange effects where needed, for temperatures as low as 2.6K. Shaul et al. \cite{Shaul2012} reported fully quantum virial coefficients of helium-4 (but without exchange) up to fourth order for temperatures of T = 2.6~K -- 1000~K.
\section{Computational details}
\section{Results and discussion}
