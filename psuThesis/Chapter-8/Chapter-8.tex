\chapter{Conclusions and Suggestions for Future Work}
    We have employed Path Integral Monte Carlo (PIMC) method to include nuclear quantum effects, Mayer Sampling Monte Carlo (MSMC) \cite{Singh2004} method for efficient evaluation of virial coefficients and some of the recently developed state-of-the-art \abinitio{} \PESs{}. The new virial coefficients for each \abinitio{} PES reported in this work are the most accurate virial coefficients for that particular system till date. These algorithms are applicable in a straightforward manner for studying other diatomics as well. Presented below are a list of original findings and contributions of this work.

    In Chapter \ref{chap:methods} we described PIMC method and explained how we can obtain quantum virial coefficients. In this process, for monatomic molecules, we also extended the conventional PIMC method with classical beads (CB) to PIMC with semi-classical beads (SCB) based approaches. Also, for diatomic molecules, we developed highly efficient, novel orientation and bond length sampling algorithms for use in PIMC method.

    In Chapter \ref{chap:he} we compared two SCB based approaches against the conventional CB based approach of the PIMC framework by computing quantum second virial coefficients of $^4$He and comparing them with literature data. In summary, we found the following order for the rate of convergence with respect to number of beads $P$: SCB-TI $>$ SCB-QFH* $>$ CB. The order for precision was found to be: SCB-TI $>$ SCB-QFH*. Compared to CB, QFH* is always worse but only marginally so; TI is almost always better and only marginally worse for a few temperatures.

    In Chapter \ref{chap:h2} we tested our novel algorithms for the diatomic molecule H$_2$ by computing quantum virial coefficients including intermolecular exchange and monomer flexibility. We found good agreement of our results with the most accurate values in literature over the wide range of temperatures considered. We observed that the performance of the algorithms, in terms of percentage of MC moves accepted, decreases with increasing $P$.

    In Chapter \ref{chap:n2} we applied our orientation algorithm in PIMC method for the diatomic molecule N$_2$ to compute new quantum virial coefficients that were previously not available in literature. We reported up to third quantum virial coefficients using CB and SCB based approaches and observed good agreement with most accurate semi-classical values in literature. Based on our findings for the case of N$_2$, we concluded that SCB based approaches did not offer much improvement compared to CB based approaches.

    In Chapter \ref{chap:o2} we reported new quantum second virial coefficients of O$_2$ using our orientation sampling algorithm and PIMC method. We found good agreement with most accurate semi-classical values in litereature.

    In Chapter \ref{chap:h2o} we reported new semi-classical third virial coefficients of H$_2$O using MSMC method and observed good agreement with most accurate classical results in literature.

    Ideas for future work include a more rigorous analysis of the algorithms that were developed in Chapter \ref{chap:methods} of this work to identify potential sources of inefficiency. This entails carrying out studies to analyze not only how each of the algorithm performs by itself but also the nature of their simultaneous presence in MC simulations. Also, a detailed comparison of the computational costs of the algorithms against conventionally used ones could provide further insight or at the very least, lead to quantification of the computational effort saved.
    
    PIMC calculations for multiatomic molecules are virtually non-existent because of the lack of sampling algorithms needed to compute high quality virial coefficient data. Just like the diatomic case, the assumption of independent atoms is expected to simplify this problem considerably. As a first step, we suggest investigation of rigid multiatomic molecules that are linear such as CO$_2$, C$_2$H$_2$ and others. Having already laid the ground work for the diatomic molecule, this seems like the next logical step towards ultimately being able to compute quantum virial coefficients for non-linear molecules. If accomplished successfully, this could lead to better quality virial coefficient data which may then be used for various applications, most notably, improvement of \abinitio{} \PESs{}. Quaternions might particularly be useful for this next step as they provide an easier route to handling 3D rotations and hence more efficient algorithms. Although their properties, functionality and application in 3D rotations is fairly well established for MC simulations (see e.g. Refs.\cite{Karney2007,Sinkovits2012}), their adaptation for this particular application is expected to be challenging.
