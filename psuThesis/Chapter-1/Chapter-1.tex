%SourceDoc ../thesis.tex
\chapter{Background \& Motivation} \label{chapter1:bgm}
    \section{Virial coefficients}
        The virial equation of state, given as:
        \begin{equation}
          \frac{p}{\rho k_\textrm{B}T} = 1 + B_2(T) \rho + B_3(T) \rho^2 + \ldots,
        \end{equation}
        expresses the pressure $p$ as a power series in the number density $\rho$ of the system; $T$ is the temperature and $k_\textrm{B}$ is the Boltzmann constant. Through this formula, the temperature-dependent virial coefficients, $B_n(T)$, can be used to estimate a variety of physical properties in addition to the pressure, such as the Joule-Thomson coefficient,  critical properties, and others. The unique feature of the virial equation of state, in comparison to other thermodynamic models, is that it represents rigorously the effect of interactions among $N$ molecules, such that if given a molecular model, the virial coefficients for its equation of state can be determined without approximation. This can be clearly seen from the analytic expressions for the virial coefficients in terms of $N$-body configurational integrals \cite{Tester} which depend only on the interaction potential between $N$ molecules as:
        \begin{equation}\label{eq: bn}
            \begin{aligned}
                B_2(T) &= - \frac{1}{2! {V}}  \Big[ Z_2^* - Z_1^{*2} \Big]\\
                B_3(T) &= - \frac{1}{3! {V}^2}  \Big[ {V}  \big( Z_3^* - 3  Z_2^*  Z_1^* + 2  Z_1^{*3} \big) - 3  {\big( Z_2^* - Z_1^{*2} \big)}^2 \Big]
            \end{aligned}
        \end{equation}
        where, $ Z_N^* \equiv N!  {\Big( \displaystyle\frac{{V}}{Q_1} \Big)}^N  Q_N ;$ is the $N$-body configurational integral, $Q_N$ is the $N$-body canonical partition function, and ${V}$ is the volume.

        The $N$-body configurational integral, which depends on the $N$-body interaction potential, becomes exponentially more difficult to compute with increasing $N$ (\hl{include the exact numbers from Hansen}). For extremely simple interaction potentials like hard spheres, up to fourth virial coefficients may be calculated analytically \cite{Masters2008}. Higher order virial coefficients using more complicated interaction potentials need to be evaluated numerically through quadrature or by using Monte Carlo (MC) simulations. Upon further simplification and assuming pairwise additivity (\hl{may be include more detailed formulae}) of the potential, we can rewrite Eq.\eqref{eq: bn} using Mayer $f$-functions as \cite{Masters2008,Hansen}
        \begin{equation} \label{eq: mayerfn}
            \begin{aligned}
                B_2(T) &= -\frac{1}{2} \displaystyle\int d1 ~ f(0,1)\\
                B_3(T) &= -\frac{1}{3} \displaystyle\int \int d1~d2~f(0,1)~f(0,2)~f(1,2)
            \end{aligned}
        \end{equation}
        where $f(0,1) = \Big( \exp \big[ -\beta U_2(\bm{r}) \big] - 1 \Big) $ and indices `1' and `2' denote the position and orientational degrees of freedom of molecules 1 and 2, respectively, with respect to molecule `0' at the origin.
    \section{Interaction potentials}
        \subsection{Empirical potentials and \emph{ab initio} potentials}
        \subsection{Classification of virial coefficients}
    \section{Research goals}
