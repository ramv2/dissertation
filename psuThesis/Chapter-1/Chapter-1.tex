%SourceDoc ../thesis.tex
\chapter{Background \& Motivation} \label{chapter1:bgm}
    \section{Virial coefficients}
        The virial equation of state, given as:
        \begin{equation}
          \frac{p}{\rho k_\textrm{B}T} = 1 + B_2(T) \rho + B_3(T) \rho^2 + \ldots,
        \end{equation}
        expresses the pressure $p$ as a power series in the number density $\rho$ of the system; $T$ is the temperature and $k_\textrm{B}$ is the Boltzmann constant. Through this formula, the temperature-dependent virial coefficients, $B_n(T)$, can be used to estimate a variety of physical properties in addition to the pressure, such as the Joule-Thomson coefficient,  critical properties, and others. The unique feature of the virial equation of state, in comparison to other thermodynamic models, is that it represents rigorously the effect of interactions among $N$ molecules, such that if given a molecular model, the virial coefficients for its equation of state can be determined without approximation. This can be clearly seen from the analytic expressions for the virial coefficients in terms of $N$-body configurational integrals \cite{Tester} which depend only on the interaction potential between $N$ molecules as:
        \begin{equation}\label{eq: bn}
            \begin{aligned}
                B_2(T) &= - \frac{1}{2! {V}}  \Big[ Z_2^* - Z_1^{*2} \Big]\\
                B_3(T) &= - \frac{1}{3! {V}^2}  \Big[ {V}  \big( Z_3^* - 3  Z_2^*  Z_1^* + 2  Z_1^{*3} \big) - 3  {\big( Z_2^* - Z_1^{*2} \big)}^2 \Big]
            \end{aligned}
        \end{equation}
        where, $ Z_N^* \equiv N!  {\Big( \displaystyle\frac{{V}}{Q_1} \Big)}^N  Q_N ;$ is the $N$-body configurational integral, $Q_N$ is the $N$-body canonical partition function, and ${V}$ is the volume.

        The $N$-body configurational integral, which depends on the $N$-body interaction potential, becomes exponentially more difficult to compute with increasing $N$ (\hl{include the exact numbers from Hansen}). For extremely simple interaction potentials like hard spheres, up to fourth virial coefficients may be calculated analytically \cite{Masters2008}. Higher order virial coefficients using more complicated interaction potentials need to be evaluated numerically through quadrature or by using Monte Carlo (MC) simulations. Upon further simplification and assuming pairwise additivity (\hl{may be include more detailed formulae}) of the potential, we can rewrite Eq.\eqref{eq: bn} using Mayer $f$-functions as \cite{Masters2008,Hansen}
        \begin{equation} \label{eq: mayerfn}
            \begin{aligned}
                B_2(T) &= -\frac{1}{2} \displaystyle\int d1 ~ f(0,1)\\
                B_3(T) &= -\frac{1}{3} \displaystyle\int \int d1~d2~f(0,1)~f(0,2)~f(1,2)
            \end{aligned}
        \end{equation}
        where $f(0,1) = \Big( \exp \big[ -\beta U_2(\bm{r}) \big] - 1 \Big) $ and indices `1' and `2' denote the position and orientational degrees of freedom of molecules 1 and 2, respectively, with respect to molecule `0' at the origin.

        Virial coefficients can be evaluated in two ways: a) computationally, using numerical methods to evaluate the configuration integrals given an input interaction potential, and b) experimentally, typically by collecting pressure-density data and regressing its behavior at the $\rho \to 0$ limit. By comparing the values determined using the above two methods, different models of the interaction potential between $N$ molecules can be ranked for their physical accuracy. 

        Empirical potentials are fit to experimental data over a broad range of conditions, and therefore they tend to describe the interaction potential as the net result of a variety of physical phenomena taking place simultaneously (including, for example, multibody interactions and nuclear quantum effects). The way that these phenomena combine to give an effective potential will depend on the state condition as well as the experimental properties being fit. Consequently, unless fit directly to them, empirical potentials often perform poorly in comparison to experimental virial coefficients \cite{Benjamin2007}, which depend purely on the interaction of a specific number of molecules.

        On the other hand, \abInitio potentials involve solving the \Schrodinger equation by using different basis functions and levels of theory and often can yield far more accurate interaction potentials as a result. Large computational resources are required to obtain coefficient with a useful level of accuracy, so the application of \abInitio method in this respect has been limited to low-order coefficients for small molecules. However, steady progress is being made \cite{Boothroyd2003,Hodges2004,Garberoglio2012,Shaul2012,Garberoglio2013,Hellmann2013,Garberoglio2014,Garberoglio2014mix,Hellmann2014,Schultz2015,Tat2015}. Almost all \abInitio potentials involve at the heart of their development, solving the electronic \Schrodinger equation using the Born-Oppenheimer approximation. Under this approximation, the kinetic energy of the nucleus is considered negligible when compared to the electrons, and therefore all the nuclei are assumed to be fixed at certain locations when solving the electronic structure. For the purpose of calculating really virial accurate coefficients using \abInitio potentials, one needs to account for nuclear quantum effects explicitly, especially at low temperatures or for light atoms, such as hydrogen. Based on the above, virial coefficients can be classified into the following three categories: 

        \begin{description}
            \item[Classical virial coefficients] \hfill \\
                The virial coefficients that are calculated from an input interaction potential (empirical or \abInitio PES) without modification.
            \item[Semi-classical virial coefficients] \hfill \\
                The virial coefficients that are calculated using an effective potential such as the Quadratic Feynman-Hibbs (QFH) \cite{Feynman} that includes a quantum correction.
            \item[Quantum virial coefficients] \hfill \\
                The virial coefficients that are calculated using an empirical potential (quantum) or \abInitio PES (fully quantum) while explicitly including nuclear quantum effects.
        \end{description}
    \section{\emph{Ab initio} Potentials}
        In this section, different quantum chemistry methods and theories that are used in the development of the \emph{ab intio} PESs are briefly mentioned. This will help develop an idea of the amount of computational effort involved in such calculations, which will later be helpful in appreciating even the subtlest improvements in efficiency of algorithms that are used in conjunction with such PESs.

        The rest of this dissertation is organized as follows: Chapter 2 provides detailed information of the existing as well as novel methods, algorithms and techniques to compute quantum virial coefficients; Chapters 3 through 7 are each dedicated for a particular system of choice (helium, hydrogen, nitrogen, oxygen and water respectively) and include background, computational details, quantum virial coefficient results and their discussion; Chapter 8 contains concluding remarks and future direction of work including suggestions for future projects. 
