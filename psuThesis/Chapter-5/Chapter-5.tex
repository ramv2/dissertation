\chapter{Nitrogen}
\label{chap:n2}
\SCBtrue
\section{Recent Developments}
    Experimental second virial coefficients have often complemented \abinitio{} \PESs{} for nitrogen that have resulted in rescaled or improved \PESs{}. With the advent of computational techniques we expect that theory is soon to replace experiments as the source of high quality virial coefficients (see e.g. Shaul et al. \cite{Shaul2012SC} for the case of helium-4). Such highly accurate and precise second virial coefficient data will lead to better quality \PESs{}, using which one can predict thermodynamic, transport, structural and other properties of nitrogen much more accurately. Berns and van der Avoird \cite{Berns1980} reported one of the first \abinitio{} PES (BvdA) for the Nitrogen (${\rm N_2 - N_2}$) dimer including first order exchange, short-range electrostatic contributions and long-range data of Mulder et al. \cite{Mulder1980}. They developed two analytic representations of their calculations: 1) A spherical harmonic expansion and 2) a two site point charge model with a shifted force center. Describing the region near the van der Waals minimum accurately, was one of the primary goals of their work. The BvdA potential was used to study crystal structure properties \cite{Luty1980} and transport properties \cite{Nyeland1984}. Whereas the long-range interaction calculations performed by Mulder et al. \cite{Mulder1980} were obtained using uncoupled HF method, Visser et al. \cite{Visser1983} performed time-dependent coupled HF method and consequently reported more accurate long-range interaction coefficients. Ad van der Avoird et al. \cite{vanDerAvoird1986} reported a new potential(AWJ) by including these improved long-range coefficients and more basis sets. However, it was observed that the van der Waal well prediction of the potential was too shallow. This resulted in large discrepancies between the calculated second virial coefficient values and the experimentally observed results. Neglect of intramolecular electron correlation leading to significant short-range exchange repulsion in the intermolecular potential was identified and later confirmed to be the possible cause for this discrepancy. Therefore, they introduced two scaling constants to change the size and slope of the short-range exchange repulsion term to obtain a rescaled potential that yielded second virial coefficients that were in much better agreement with experiments. Highly accurate experimental virial coefficient data sets were used to improve \abinitio{} PES obtained as a result of calculations performed at the CCSD(T) level with large basis sets by Stallcop et al. \cite{Stallcop1997}. A semi-empirical relation between the parameters of the repulsive potential and the accurate virial coefficient data sets was also reported. Cappelletti et al. \cite{Cappelletti1998} also reported a potential which was an improvement of the AWJ potential, by firstly computing \abinitio{} PES at the CCSD(T) level of theory and then using accurate data from scattering, virial coefficients and other property measurements to improve the PES quality. An extensive \abinitio{} study was conducted by Leonhard et al. \cite{Leonhard2002} to not only develop a improved 2-body potential but also to analyze two 3-body potentials for their capability to reproduce thermodynamic properties if both the 2-body as well as 3-body potentials were used. The 2-body \abinitio{} PES was rescaled to fit accurate second virial coefficient data. After comparison to experimental results, the 3-body potential due to Axilrod Teller (AT) \cite{Axilrod1943} performed better than the one due to Stogryn \cite{Stogryn1969}, when used in combination with the 2-body potential. Gomez et al. \cite{Gomez2007} reported an \abinitio{} PES that was calculated using SAPT. The accuracy of perturbative energies were compared against calculations at the CCSD(T) level of theory using large basis sets. They observed better agreement to experimental virial coefficients than previous unscaled \PESs{} for the range of temperatures investigated. They attributed the consistently lower second virial coefficient predictions by using their PES, to possible overestimation of the van der Waal's well within their calculations. Str\k{a}k et al. \cite{Strak2007} also performed \abinitio{} calculations at the CCSD(T) level of theory that resulted in a potential that was used in combination with Molecular Dynamics (MD) simulations to obtain an equation of state for the dimer. The resulting predictions of physical properties based on this equation of state were found to be in excellent agreement with experiments for a wide range of temperatures (up to 2000K) and pressures (up to 30GPa). Hellmann \cite{Hellmann2013} recently performed \abinitio{} calculations at the CCSD(T) level of theory using basis sets up to aug-cc-pV5Z supplemented with bond functions. The resulting 2-body (scaled) and 3-body potential was used to compute second and third (including 3-body effects) virial coefficients that were most accurate with experiments than previously reported values.
\section{Computational Details}
    \subsection{Second Virial Coefficients}
    \label{subsubsec:b2n2}
        The \abinitio{} potential (denoted as $U_2^H$) that we used was due to Hellmann \cite{Hellmann2013}. We have performed classical and semi-classical virial coefficient calculations, each using $10^9$ configuration samples and two types of MC trials 1) translation and 2) rotation of the dimer. We have used the semi-classical QFH version of the potential which is given as follows \cite{Feynman,Schenter2002}:
        \begin{equation}
        \label{eq:QFH}
            U_2^{\rm QFH} = U_2^H + \displaystyle\frac{\hbar^2 \beta}{24 m} \Big[ \pmb{\nabla}^2 U_2^H + \frac{2}{r} \pmb{\nabla} U_2^H \Big].
        \end{equation}

        We note that whereas Hellmann reports first order quantum corrections to the virial coefficients using Pack's \cite{Pack1983} formulation, we have employed QFH version of $U_2^H$ (Eq. \eqref{eq:QFH}). Therefore we expect significant differences between our virial coefficient values for the semi-classical calculations and Hellmann's results for the same, especially at low temperatures. We have performed PIMC calculations using $P$ = 8, 16, 32 and 64 classical images for each of the temperatures reported by Hellmann and collected $10^8$ configuration samples each. Apart from the translation of the dimer, we also include MC trials to regrow the ring by translating the beads (see Sec. \ref{subsec:orMove}) and our recently developed orientation algorithm \cite{hydrogen} (explained in Sec. \ref{subsec:orMove}). We have performed PIMC calculations using the semi-classical beads (SCB) approach developed in Ref. \cite{Fomms2016} and also the exact same conditions as PIMC with classical beads (CB) that are mentioned above. We note that for the PI-SCB calculations, we employed the TI propagator (Eq. \eqref{eq:TI}).

    \subsection{Third Virial Coefficients}
        We repeated all the calculations under the same conditions (except the semi-classical additive as well as non-additive calculations for which we used $10^8$ configuration samples as opposed to $10^9$ for the second virial coefficient case) that were explained in Sec. \ref{subsubsec:b2n2} for computing the additive as well as non-additive contributions to the third virial coefficients. Note that we employed the non-additive 3-body potential that was also reported by Hellmann \cite{Hellmann2013} to compute the non-additive contributions. Also, we performed the semi-classical additive as well as non-additive simulations using the same QFH version of $U_2^H$ as in Eq. \eqref{eq:QFH} whereas Hellmann derived a new expression for the first order quantum correction to the third virial coefficient. Therefore, like the second virial coefficient, we expect significant differences between our semi-classical results and Hellmann's, especially at low temperatures. We note that all the third virial coefficient simulations were performed at a set of 90 temperatures considered by Hellmann. We have also performed PIMC calculations with CB and SCB approaches similar to the second virial coefficient case and the same set of temperatures.
\section{Results and Discussion}
    \subsection{Second Virial Coefficients}
        In order to reproduce the results of Hellmann and to ensure that our implementation of the \abinitio{} potential was correct, we performed classical virial coefficient calculations as a first step. In Fig. \ref{fig:B2CLN2} we show the difference, between our results and Hellmann's for the second virial coefficient as a function of temperature, in the main plot and show the actual values in the inset plot, also as a function of temperature.
        \begin{figure}[!htbp]
            \centering
            \includegraphics[scale=0.20,keepaspectratio]{Chapter-5/Figures/B2CL9sResults.png}
            \caption{Classical second virial coefficient ($B_2$) values for N$_2$ compared against Hellmann's values \cite{Hellmann2013}. Error bars represent one standard deviation of the mean (68\% confidence interval). }
            \label{fig:B2CLN2}
        \end{figure}
        It can be clearly seen from Fig. \ref{fig:B2CLN2} that our classical results are in excellent statistical agreement with Hellmann's for the entire range of temperatures, suggesting that our implementation of the \abinitio{} potential is correct.

        In Fig. \ref{fig:B2SCN2} we show the difference between, our semi-classical second virial coefficient results and Hellmann's as a function of temperature, in the main plot and show the actual values in the inset plot, also as a function of temperature.
        \begin{figure}[!htbp]
            \centering
            \includegraphics[scale=0.20,keepaspectratio]{Chapter-5/Figures/B2N2SC9sResults.png}
            \caption{Semi-classical second virial coefficient ($B_2$) values for N$_2$ compared against Hellmann's values \cite{Hellmann2013}. Error bars represent one standard deviation of the mean (68\% confidence interval). Our low temperature results are off the bottom of the scale.}
            \label{fig:B2SCN2}
        \end{figure}
        From Fig. \ref{fig:B2SCN2} and the argument made in Sec. \ref{subsubsec:b2n2}, we can clearly see that our semi-classical results are in excellent agreement with Hellmann's for temperatures $T \ge 100$K. The deviation observed for lower temperatures can be explained by the fact that whereas we have used the semi-classical QFH version (Eq. \eqref{eq:QFH}) of the interaction potential, Hellmann computed first order quantum corrections using Pack's \cite{Pack1983} formulation. From Fig. \ref{fig:B2SCN2} one can also make the observation that the semi-classical QFH results and Hellmann's results agree well for $T \ge 100$K below which we start noticing significant deviations between the two.

        In Figs. \ref{fig:B2N2AllCLSCDiff}, \ref{fig:B2N2AllPIDiff} and \ref{fig:B2N2AllSCBDiff} we show second virial coefficient values for N$_2$ computed using classical \& semi-classical, PI-CB and the PI-SCB approaches respectively. Under the reasonable assumption that the PI-SCB approach with $P$ = 64 yields the most accurate virial coefficient value, we compare the values computed using the classical, semi-classical and PI-CB approaches with the $P$ = 64 PI-SCB values. In Fig. \ref{fig:B2N2AllCLSCDiff}, we can see that the classical values start to become statistically different from the semi-classical as well as PI results around $T \approx 340$K, with the magnitude of the difference increasing with decreasing temperature. Similarly, the semi-classical results of Hellmann start to deviate around $T \approx 60$K, with the magnitude of the difference increasing for lower temperatures. Our semi-classical results seem to be in agreement with our PI-SCB results for the entire range of temperatures assuming that the agreement for some temperatures are poor due to random noise. Although this needs further confirmation, given that our classical and semi-classical results agree well with Hellmann's results (Figs. \ref{fig:B2CLN2} and \ref{fig:B2SCN2}), this assumption seems reasonable. Ignoring random noise, we observe that our PI-CB values (for all $P$) and PI-SCB values (for $P$ = 8, 16 and 32) are also in good agreement with the PI-SCB $P$ = 64 values in Figs. \ref{fig:B2N2AllPIDiff} and \ref{fig:B2N2AllSCBDiff} respectively. This suggests that for the range of temperatures considered, nuclear quantum effects are accurately captured by the semi-classical treatment using the QFH effective potential.

        \begin{figure}[!htbp]
            \centering
            \includegraphics[scale=0.20,keepaspectratio]{Chapter-5/Figures/B2N2AllCLSCDiff.png}
            \caption{Second virial coefficients of N$_2$ computed using PI-SCB approach with $P$ = 64 images compared against results using classical (cl) and semi-classical (sc) approaches. Main figure is the difference between the PI-SCB ($P$ = 64) values and the classical as well as semi-classical vales, while the inset shows the coefficients before differencing. Error bars represent one standard deviation of the mean (68\% confidence interval). Low temperature values for classical and semi-classical calculations are off the bottom of the scale.}
            \label{fig:B2N2AllCLSCDiff}
        \end{figure}

        \begin{figure}[!htbp]
            \centering
            \includegraphics[scale=0.20,keepaspectratio]{Chapter-5/Figures/B2N2AllPIDiff.png}
            \caption{Second virial coefficients of N$_2$ computed using PI-SCB approach with $P$ = 64 images compared against PI-CB approach for different values of $P$. Main figure is the difference between the PI-SCB ($P$ = 64) values and the CB vales, while the inset shows the coefficients before differencing. Error bars represent one standard deviation of the mean (68\% confidence interval).}
            \label{fig:B2N2AllPIDiff}
        \end{figure}

        \begin{figure}[!htbp]
            \centering
            \includegraphics[scale=0.20,keepaspectratio]{Chapter-5/Figures/B2N2AllSCBDiff.png}
            \caption{Second virial coefficients of N$_2$ computed using PI-SCB approach for different values of $P$. Main figure is the difference between the $P$ = 64 values and lower $P$ values, while the inset shows the coefficients before differencing. Error bars represent one standard deviation of the mean (68\% confidence interval).}
            \label{fig:B2N2AllSCBDiff}
        \end{figure}

    \subsection{Third Virial Coefficients}
        In Figs. \ref{fig:B3CLN2} and \ref{fig:B3SCN2}, we show the total classical and semi-classical third virial coefficients respectively of N$_2$, i.e., including additive and non-additive contributions.
        \begin{figure}[!htbp]
            \centering
            \includegraphics[scale=0.20,keepaspectratio]{Chapter-5/Figures/B3N2CL9sResults.png}
            \caption{Total classical second third coefficient ($B_3$) values of N$_2$ (i.e., including additive as well as non-additive contributions) compared against Hellmann's values \cite{Hellmann2013}. Error bars represent one standard deviation of the mean (68\% confidence interval).}
            \label{fig:B3CLN2}
        \end{figure}

        \begin{figure}[!htbp]
            \centering
            \includegraphics[scale=0.20,keepaspectratio]{Chapter-5/Figures/B3N2SC9sResults.png}
            \caption{Semi-classical third virial coefficient ($B_3$) values of N$_2$ (i.e., including additive as well as non-additive contributions) compared against Hellmann's values \cite{Hellmann2013}. Error bars represent one standard deviation of the mean (68\% confidence interval). Our low temperature values are off the bottom of the scale.}
            \label{fig:B3SCN2}
        \end{figure}

        From Fig. \ref{fig:B3CLN2}, we see that our classical $B_3$ results agree quite well with Hellmann's for the entire range of temperatures considered (except for a few temperatures assuming random noise). From Fig. \ref{fig:B3SCN2}, we can see that there is quite a bit of deviation of our semi-classical results from Hellmann's corresponding results for $T \le 300$K. As explained earlier, this might be attributed to the use of a different semi-classical form of the \abinitio{} potential. It can also be seen that the deviations become less significant for $T > 300$K.

        In Figs. \ref{fig:B3N2AllCLSCDiff}, \ref{fig:B3N2AllPIDiff} and \ref{fig:B3N2AllSCBDiff} we show total third virial coefficient values (including additive as well as non-additive contributions) for N$_2$ computed using classical \& semi-classical, PI-CB and the PI-SCB approaches respectively. Under the reasonable assumption that the PI-SCB approach with $P$ = 64 yields the most accurate virial coefficient value, we compare the values computed using the classical, semi-classical and PI-CB approaches with the $P$ = 64 PI-SCB values. In Fig. \ref{fig:B3N2AllCLSCDiff}, we can see that the classical values start to become statistically different from the semi-classical as well as PI results around $T \approx 110$K with the magnitude of the difference increasing with decreasing temperature. Similarly, the semi-classical results of Hellmann start to deviate around $T \approx 120$K, with the magnitude of the difference increasing for lower temperatures. Our semi-classical results seem to be in agreement with our PI-SCB results for the entire range of temperatures assuming that the agreement for some temperatures are poor due to random noise. Although this needs further confirmation, given that our classical and semi-classical results agree well with Hellmann's results (Figs. \ref{fig:B3CLN2} and \ref{fig:B3SCN2}), this assumption seems reasonable. Ignoring random noise, we observe that our PI-CB values (for all $P$) and PI-SCB values (for $P$ = 8, 16 and 32) are also in good agreement with the PI-SCB $P$ = 64 values in Figs. \ref{fig:B3N2AllPIDiff} and \ref{fig:B3N2AllSCBDiff} respectively. This suggests that for the range of temperatures considered, nuclear quantum effects are accurately captured by the semi-classical treatment using the QFH effective potential.
        \begin{figure}[!htbp]
            \centering
            \includegraphics[scale=0.20,keepaspectratio]{Chapter-5/Figures/B3N2AllCLSCDiff.png}
            \caption{Difference between the total third virial coefficients of N$_2$ computed using PI-SCB approach with $P$ = 64 images and classical (cl),semi-classical (sc) approaches. Error bars represent one standard deviation of the mean (68\% confidence interval). Low temperature values for classical and semi-classical calculations are off the bottom of the scale.}
            \label{fig:B3N2AllCLSCDiff}
        \end{figure}
        \begin{figure}[!htbp]
            \centering
            \includegraphics[scale=0.20,keepaspectratio]{Chapter-5/Figures/B3N2AllPIDiff.png}
            \caption{Difference between the total third virial coefficients of N$_2$ computed using PI-SCB approach with $P$ = 64 images and results using classical (cl),semi-classical (sc) approaches. The scale has been purposely set so as to exemplify the nature of the oscillations observed. Error bars represent one standard deviation of the mean (68\% confidence interval). Low temperature values for classical and semi-classical calculations are off the bottom of the scale.}
            \label{fig:B3N2AllPIDiff}
        \end{figure}
        \begin{figure}[!htbp]
            \centering
            \includegraphics[scale=0.20,keepaspectratio]{Chapter-5/Figures/B3N2AllSCBDiff.png}
            \caption{Difference between the total third virial coefficients of N$_2$ computed using PI-SCB approach with $P$ = 64 images and classical (cl),semi-classical (sc) approaches. The scale has been purposely set so as to exemplify the nature of the oscillations observed. Error bars represent one standard deviation of the mean (68\% confidence interval). Low temperature values for classical and semi-classical calculations are off the bottom of the scale.}
            \label{fig:B3N2AllSCBDiff}
        \end{figure}

    \section{Conclusions}
    \label{sec:conclusion}
        We have calculated fully quantum virial coefficients for nitrogen dimers by explicitly including nuclear quantum effects via PIMC method. Additionally, we have calculated fully quantum virial coefficients for the nitrogen dimer using PIMC with semi-classical beads, an approach that was recently \cite{Fomms2016} introduced. We have also employed a new algorithm that we recently \cite{hydrogen} developed to sample orientations of diatomic molecules, in all our PIMC calculations. We have compared our PIMC results against the most accurate data available in literature and found overall good agreement for the semi-classical as well as PIMC results with CB and SCB approaches. Based on the observations made, we conclude that for the range of temperatures considered, the semi-classical QFH results are accurate enough to capture nuclear quantum effects. Investigations of lower temperatures ($T \le 50$K) could necessitate the use of PIMC methods, especially the SCB approaches. As a result of this work, the validity and robustness of the orientation sampling algorithm \cite{hydrogen} have been successfully tested.

    \section{Tabulated Results}
    \label{sec:chap5-tables}
        We report tabulated results for the virial coefficients of nitrogen in this section.
        \pgfplotstableset{
            begin table=\begin{longtable},
            end table=\end{longtable},
        }
        \pgfplotstabletypeset[
            columns/temperature/.style={
                string type,
                column name={Temperature(\si{\kelvin})}
            },
            columns/b2sc/.style={
                string type,
                column name=B$_2$(\si{cm^3/mol})
            },
            columns/b3sca/.style={
                string type,
                column name=B$^\text{A}_3$(\si{cm^6/mol^2})
            },
            columns/b3scna/.style={
                string type,
                column name=B$^\text{NA}_3$(\si{cm^6/mol^2})
            },
            columns/b3sctot/.style={
                string type,
                column name=B$^\text{Tot}_3$(\si{cm^6/mol^2})
            },
            every head row/.append style={before row={\caption{Second and third virial coefficients of nitrogen computed using the semi-classical QFH effective potential (Eq. \eqref{eq:QFH})and the non-additive three body potential of Hellmann\cite{Hellmann2013}. B$^\text{A}_3$ and B$^\text{NA}_3$ represent the additive and non-additive contributions respectively, to the third virial coefficient. B$^\text{Tot}_3$ represents the full third virial coefficient value obtained by summing the additive and non-additive contributions. Values in parentheses are standard uncertainties in the rightmost digit(s).}\label{tab:9sSCResults}\\\toprule},after row=\midrule\endfirsthead},
            every first row/.append style={before row={\multicolumn{5}{c}{\tablename\ \thetable{} -- Continued from previous page.}\\\toprule},after row=\midrule\endhead},
            every last row/.style={after row=\bottomrule},
        ]{Chapter-5/Tables/9sSCResultsTable.dat}

        \pgfplotstabletypeset[
            columns/temperature/.style={
                string type,
                column name={Temperature(\si{\kelvin})}
            },
            columns/b2cb8/.style={
                string type,
                column name=B$^{8}_2$(\si{cm^3/mol})
            },
            columns/b2cb16/.style={
                string type,
                column name=B$^{16}_2$(\si{cm^3/mol})
            },
            columns/b2cb32/.style={
                string type,
                column name=B$^{32}_2$(\si{cm^3/mol})
            },
            columns/b2cb64/.style={
                string type,
                column name=B$^{64}_2$(\si{cm^3/mol})
            },
            every head row/.append style={before row={\caption{Second virial coefficient of nitrogen computed using PIMC method and CB based approach. Here B$^P_2$ represents the second virial coefficient results using $P$ beads or images per ring. Values in parentheses are standard uncertainties in the rightmost digit(s).}\label{tab:8sB2PICBResults}\\\toprule},after row=\midrule\endfirsthead},
            every first row/.append style={before row={\multicolumn{5}{c}{\tablename\ \thetable{} -- Continued from previous page.}\\\toprule},after row=\midrule\endhead},
            every last row/.style={after row=\bottomrule},
        ]{Chapter-5/Tables/8sB2PICBResultsTable.dat}

        \pgfplotstabletypeset[
            columns/temperature/.style={
                string type,
                column name={Temperature(\si{\kelvin})}
            },
            columns/b2scb8/.style={
                string type,
                column name=B$^{8}_2$(\si{cm^3/mol})
            },
            columns/b2scb16/.style={
                string type,
                column name=B$^{16}_2$(\si{cm^3/mol})
            },
            columns/b2scb32/.style={
                string type,
                column name=B$^{32}_2$(\si{cm^3/mol})
            },
            columns/b2scb64/.style={
                string type,
                column name=B$^{64}_2$(\si{cm^3/mol})
            },
            every head row/.append style={before row={\caption{Second virial coefficient of nitrogen computed using PIMC method and SCB based approach. Here B$^P_2$ represents the second virial coefficient results using $P$ beads or images per ring. Values in parentheses are standard uncertainties in the rightmost digit(s).}\label{tab:8sB2PISCBResults}\\\toprule},after row=\midrule\endfirsthead},
            every first row/.append style={before row={\multicolumn{5}{c}{\tablename\ \thetable{} -- Continued from previous page.}\\\toprule},after row=\midrule\endhead},
            every last row/.style={after row=\bottomrule},
        ]{Chapter-5/Tables/8sB2PISCBResultsTable.dat}

        \pgfplotstabletypeset[
            columns/temperature/.style={
                string type,
                column name={Temperature(\si{\kelvin})}
            },
            columns/b3cb8a/.style={
                string type,
                column name=B$^\text{A}_3$(\si{cm^6/mol^2})
            },
            columns/b3cb8na/.style={
                string type,
                column name=B$^\text{NA}_3$(\si{cm^6/mol^2})
            },
            columns/b3cb8tot/.style={
                string type,
                column name=B$^\text{Tot}_3$(\si{cm^6/mol^2})
            },
            every head row/.append style={before row={\caption{Third virial coefficients of nitrogen computed using the semi-classical QFH effective potential (Eq. \eqref{eq:QFH}), the non-additive three body potential of Hellmann\cite{Hellmann2013}, PIMC method with $P$ = 8 and CB approach. B$^\text{A}_3$ and B$^\text{NA}_3$ represent the additive and non-additive contributions respectively, to the third virial coefficient. B$^\text{Tot}_3$ represents the full third virial coefficient value obtained by summing the additive and non-additive contributions. Values in parentheses are standard uncertainties in the rightmost digit(s).}\label{tab:8nb8sPICBResults}\\\toprule},after row=\midrule\endfirsthead},
            every first row/.append style={before row={\multicolumn{4}{c}{\tablename\ \thetable{} -- Continued from previous page.}\\\toprule},after row=\midrule\endhead},
            every last row/.style={after row=\bottomrule},
        ]{Chapter-5/Tables/8nb8sB3PICBResultsTable.dat}

        \pgfplotstabletypeset[
            columns/temperature/.style={
                string type,
                column name={Temperature(\si{\kelvin})}
            },
            columns/b3cb8a/.style={
                string type,
                column name=B$^\text{A}_3$(\si{cm^6/mol^2})
            },
            columns/b3cb8na/.style={
                string type,
                column name=B$^\text{NA}_3$(\si{cm^6/mol^2})
            },
            columns/b3cb8tot/.style={
                string type,
                column name=B$^\text{Tot}_3$(\si{cm^6/mol^2})
            },
            every head row/.append style={before row={\caption{Third virial coefficients of nitrogen computed using the semi-classical QFH effective potential (Eq. \eqref{eq:QFH}), the non-additive three body potential of Hellmann\cite{Hellmann2013}, PIMC method with $P$ = 16 and CB approach. B$^\text{A}_3$ and B$^\text{NA}_3$ represent the additive and non-additive contributions respectively, to the third virial coefficient. B$^\text{Tot}_3$ represents the full third virial coefficient value obtained by summing the additive and non-additive contributions. Values in parentheses are standard uncertainties in the rightmost digit(s).}\label{tab:16nb8sPICBResults}\\\toprule},after row=\midrule\endfirsthead},
            every first row/.append style={before row={\multicolumn{4}{c}{\tablename\ \thetable{} -- Continued from previous page.}\\\toprule},after row=\midrule\endhead},
            every last row/.style={after row=\bottomrule},
        ]{Chapter-5/Tables/16nb8sB3PICBResultsTable.dat}

        \pgfplotstabletypeset[
            columns/temperature/.style={
                string type,
                column name={Temperature(\si{\kelvin})}
            },
            columns/b3cb8a/.style={
                string type,
                column name=B$^\text{A}_3$(\si{cm^6/mol^2})
            },
            columns/b3cb8na/.style={
                string type,
                column name=B$^\text{NA}_3$(\si{cm^6/mol^2})
            },
            columns/b3cb8tot/.style={
                string type,
                column name=B$^\text{Tot}_3$(\si{cm^6/mol^2})
            },
            every head row/.append style={before row={\caption{Third virial coefficients of nitrogen computed using the semi-classical QFH effective potential (Eq. \eqref{eq:QFH}), the non-additive three body potential of Hellmann\cite{Hellmann2013}, PIMC method with $P$ = 32 and CB approach. B$^\text{A}_3$ and B$^\text{NA}_3$ represent the additive and non-additive contributions respectively, to the third virial coefficient. B$^\text{Tot}_3$ represents the full third virial coefficient value obtained by summing the additive and non-additive contributions. Values in parentheses are standard uncertainties in the rightmost digit(s).}\label{tab:32nb8sPICBResults}\\\toprule},after row=\midrule\endfirsthead},
            every first row/.append style={before row={\multicolumn{4}{c}{\tablename\ \thetable{} -- Continued from previous page.}\\\toprule},after row=\midrule\endhead},
            every last row/.style={after row=\bottomrule},
        ]{Chapter-5/Tables/32nb8sB3PICBResultsTable.dat}

        \pgfplotstabletypeset[
            columns/temperature/.style={
                string type,
                column name={Temperature(\si{\kelvin})}
            },
            columns/b3cb8a/.style={
                string type,
                column name=B$^\text{A}_3$(\si{cm^6/mol^2})
            },
            columns/b3cb8na/.style={
                string type,
                column name=B$^\text{NA}_3$(\si{cm^6/mol^2})
            },
            columns/b3cb8tot/.style={
                string type,
                column name=B$^\text{Tot}_3$(\si{cm^6/mol^2})
            },
            every head row/.append style={before row={\caption{Third virial coefficients of nitrogen computed using the semi-classical QFH effective potential (Eq. \eqref{eq:QFH}), the non-additive three body potential of Hellmann\cite{Hellmann2013}, PIMC method with $P$ = 64 and CB approach. B$^\text{A}_3$ and B$^\text{NA}_3$ represent the additive and non-additive contributions respectively, to the third virial coefficient. B$^\text{Tot}_3$ represents the full third virial coefficient value obtained by summing the additive and non-additive contributions. Values in parentheses are standard uncertainties in the rightmost digit(s).}\label{tab:64nb8sPICBResults}\\\toprule},after row=\midrule\endfirsthead},
            every first row/.append style={before row={\multicolumn{4}{c}{\tablename\ \thetable{} -- Continued from previous page.}\\\toprule},after row=\midrule\endhead},
            every last row/.style={after row=\bottomrule},
        ]{Chapter-5/Tables/64nb8sB3PICBResultsTable.dat}

        \pgfplotstabletypeset[
            columns/temperature/.style={
                string type,
                column name={Temperature(\si{\kelvin})}
            },
            columns/b3scb8a/.style={
                string type,
                column name=B$^\text{A}_3$(\si{cm^6/mol^2})
            },
            columns/b3scb8na/.style={
                string type,
                column name=B$^\text{NA}_3$(\si{cm^6/mol^2})
            },
            columns/b3scb8tot/.style={
                string type,
                column name=B$^\text{Tot}_3$(\si{cm^6/mol^2})
            },
            every head row/.append style={before row={\caption{Third virial coefficients of nitrogen computed using the semi-classical QFH effective potential (Eq. \eqref{eq:QFH}), the non-additive three body potential of Hellmann\cite{Hellmann2013}, PIMC method with $P$ = 8 and SCB approach. B$^\text{A}_3$ and B$^\text{NA}_3$ represent the additive and non-additive contributions respectively, to the third virial coefficient. B$^\text{Tot}_3$ represents the full third virial coefficient value obtained by summing the additive and non-additive contributions. Values in parentheses are standard uncertainties in the rightmost digit(s).}\label{tab:8nb7sPISCBResults}\\\toprule},after row=\midrule\endfirsthead},
            every first row/.append style={before row={\multicolumn{4}{c}{\tablename\ \thetable{} -- Continued from previous page.}\\\toprule},after row=\midrule\endhead},
            every last row/.style={after row=\bottomrule},
        ]{Chapter-5/Tables/8nb7sB3PISCBResultsTable.dat}

        \pgfplotstabletypeset[
            columns/temperature/.style={
                string type,
                column name={Temperature(\si{\kelvin})}
            },
            columns/b3scb8a/.style={
                string type,
                column name=B$^\text{A}_3$(\si{cm^6/mol^2})
            },
            columns/b3scb8na/.style={
                string type,
                column name=B$^\text{NA}_3$(\si{cm^6/mol^2})
            },
            columns/b3scb8tot/.style={
                string type,
                column name=B$^\text{Tot}_3$(\si{cm^6/mol^2})
            },
            every head row/.append style={before row={\caption{Third virial coefficients of nitrogen computed using the semi-classical QFH effective potential (Eq. \eqref{eq:QFH}), the non-additive three body potential of Hellmann\cite{Hellmann2013}, PIMC method with $P$ = 16 and SCB approach. B$^\text{A}_3$ and B$^\text{NA}_3$ represent the additive and non-additive contributions respectively, to the third virial coefficient. B$^\text{Tot}_3$ represents the full third virial coefficient value obtained by summing the additive and non-additive contributions. Values in parentheses are standard uncertainties in the rightmost digit(s).}\label{tab:16nb7sPISCBResults}\\\toprule},after row=\midrule\endfirsthead},
            every first row/.append style={before row={\multicolumn{4}{c}{\tablename\ \thetable{} -- Continued from previous page.}\\\toprule},after row=\midrule\endhead},
            every last row/.style={after row=\bottomrule},
        ]{Chapter-5/Tables/16nb7sB3PISCBResultsTable.dat}

        \pgfplotstabletypeset[
            columns/temperature/.style={
                string type,
                column name={Temperature(\si{\kelvin})}
            },
            columns/b3scb8a/.style={
                string type,
                column name=B$^\text{A}_3$(\si{cm^6/mol^2})
            },
            columns/b3scb8na/.style={
                string type,
                column name=B$^\text{NA}_3$(\si{cm^6/mol^2})
            },
            columns/b3scb8tot/.style={
                string type,
                column name=B$^\text{Tot}_3$(\si{cm^6/mol^2})
            },
            every head row/.append style={before row={\caption{Third virial coefficients of nitrogen computed using the semi-classical QFH effective potential (Eq. \eqref{eq:QFH}), the non-additive three body potential of Hellmann\cite{Hellmann2013}, PIMC method with $P$ = 32 and SCB approach. B$^\text{A}_3$ and B$^\text{NA}_3$ represent the additive and non-additive contributions respectively, to the third virial coefficient. B$^\text{Tot}_3$ represents the full third virial coefficient value obtained by summing the additive and non-additive contributions. Values in parentheses are standard uncertainties in the rightmost digit(s).}\label{tab:32nb7sPISCBResults}\\\toprule},after row=\midrule\endfirsthead},
            every first row/.append style={before row={\multicolumn{4}{c}{\tablename\ \thetable{} -- Continued from previous page.}\\\toprule},after row=\midrule\endhead},
            every last row/.style={after row=\bottomrule},
        ]{Chapter-5/Tables/32nb7sB3PISCBResultsTable.dat}

        \pgfplotstabletypeset[
            columns/temperature/.style={
                string type,
                column name={Temperature(\si{\kelvin})}
            },
            columns/b3scb8a/.style={
                string type,
                column name=B$^\text{A}_3$(\si{cm^6/mol^2})
            },
            columns/b3scb8na/.style={
                string type,
                column name=B$^\text{NA}_3$(\si{cm^6/mol^2})
            },
            columns/b3scb8tot/.style={
                string type,
                column name=B$^\text{Tot}_3$(\si{cm^6/mol^2})
            },
            every head row/.append style={before row={\caption{Third virial coefficients of nitrogen computed using the semi-classical QFH effective potential (Eq. \eqref{eq:QFH}), the non-additive three body potential of Hellmann\cite{Hellmann2013}, PIMC method with $P$ = 64 and SCB approach. B$^\text{A}_3$ and B$^\text{NA}_3$ represent the additive and non-additive contributions respectively, to the third virial coefficient. B$^\text{Tot}_3$ represents the full third virial coefficient value obtained by summing the additive and non-additive contributions. Values in parentheses are standard uncertainties in the rightmost digit(s).}\label{tab:64nb7sPISCBResults}\\\toprule},after row=\midrule\endfirsthead},
            every first row/.append style={before row={\multicolumn{4}{c}{\tablename\ \thetable{} -- Continued from previous page.}\\\toprule},after row=\midrule\endhead},
            every last row/.style={after row=\bottomrule},
        ]{Chapter-5/Tables/64nb7sB3PISCBResultsTable.dat}
